\PassOptionsToPackage{unicode=true}{hyperref} % options for packages loaded elsewhere
\PassOptionsToPackage{hyphens}{url}
%
\documentclass[
]{article}
\usepackage{lmodern}
\usepackage{amssymb,amsmath}
\usepackage{ifxetex,ifluatex}
\ifnum 0\ifxetex 1\fi\ifluatex 1\fi=0 % if pdftex
  \usepackage[T1]{fontenc}
  \usepackage[utf8]{inputenc}
  \usepackage{textcomp} % provides euro and other symbols
\else % if luatex or xelatex
  \usepackage{unicode-math}
  \defaultfontfeatures{Scale=MatchLowercase}
  \defaultfontfeatures[\rmfamily]{Ligatures=TeX,Scale=1}
\fi
% use upquote if available, for straight quotes in verbatim environments
\IfFileExists{upquote.sty}{\usepackage{upquote}}{}
\IfFileExists{microtype.sty}{% use microtype if available
  \usepackage[]{microtype}
  \UseMicrotypeSet[protrusion]{basicmath} % disable protrusion for tt fonts
}{}
\makeatletter
\@ifundefined{KOMAClassName}{% if non-KOMA class
  \IfFileExists{parskip.sty}{%
    \usepackage{parskip}
  }{% else
    \setlength{\parindent}{0pt}
    \setlength{\parskip}{6pt plus 2pt minus 1pt}}
}{% if KOMA class
  \KOMAoptions{parskip=half}}
\makeatother
\usepackage{xcolor}
\IfFileExists{xurl.sty}{\usepackage{xurl}}{} % add URL line breaks if available
\IfFileExists{bookmark.sty}{\usepackage{bookmark}}{\usepackage{hyperref}}
\hypersetup{
  pdfborder={0 0 0},
  breaklinks=true}
\urlstyle{same}  % don't use monospace font for urls
\usepackage[margin=1in]{geometry}
\usepackage{color}
\usepackage{fancyvrb}
\newcommand{\VerbBar}{|}
\newcommand{\VERB}{\Verb[commandchars=\\\{\}]}
\DefineVerbatimEnvironment{Highlighting}{Verbatim}{commandchars=\\\{\}}
% Add ',fontsize=\small' for more characters per line
\usepackage{framed}
\definecolor{shadecolor}{RGB}{248,248,248}
\newenvironment{Shaded}{\begin{snugshade}}{\end{snugshade}}
\newcommand{\AlertTok}[1]{\textcolor[rgb]{0.94,0.16,0.16}{#1}}
\newcommand{\AnnotationTok}[1]{\textcolor[rgb]{0.56,0.35,0.01}{\textbf{\textit{#1}}}}
\newcommand{\AttributeTok}[1]{\textcolor[rgb]{0.77,0.63,0.00}{#1}}
\newcommand{\BaseNTok}[1]{\textcolor[rgb]{0.00,0.00,0.81}{#1}}
\newcommand{\BuiltInTok}[1]{#1}
\newcommand{\CharTok}[1]{\textcolor[rgb]{0.31,0.60,0.02}{#1}}
\newcommand{\CommentTok}[1]{\textcolor[rgb]{0.56,0.35,0.01}{\textit{#1}}}
\newcommand{\CommentVarTok}[1]{\textcolor[rgb]{0.56,0.35,0.01}{\textbf{\textit{#1}}}}
\newcommand{\ConstantTok}[1]{\textcolor[rgb]{0.00,0.00,0.00}{#1}}
\newcommand{\ControlFlowTok}[1]{\textcolor[rgb]{0.13,0.29,0.53}{\textbf{#1}}}
\newcommand{\DataTypeTok}[1]{\textcolor[rgb]{0.13,0.29,0.53}{#1}}
\newcommand{\DecValTok}[1]{\textcolor[rgb]{0.00,0.00,0.81}{#1}}
\newcommand{\DocumentationTok}[1]{\textcolor[rgb]{0.56,0.35,0.01}{\textbf{\textit{#1}}}}
\newcommand{\ErrorTok}[1]{\textcolor[rgb]{0.64,0.00,0.00}{\textbf{#1}}}
\newcommand{\ExtensionTok}[1]{#1}
\newcommand{\FloatTok}[1]{\textcolor[rgb]{0.00,0.00,0.81}{#1}}
\newcommand{\FunctionTok}[1]{\textcolor[rgb]{0.00,0.00,0.00}{#1}}
\newcommand{\ImportTok}[1]{#1}
\newcommand{\InformationTok}[1]{\textcolor[rgb]{0.56,0.35,0.01}{\textbf{\textit{#1}}}}
\newcommand{\KeywordTok}[1]{\textcolor[rgb]{0.13,0.29,0.53}{\textbf{#1}}}
\newcommand{\NormalTok}[1]{#1}
\newcommand{\OperatorTok}[1]{\textcolor[rgb]{0.81,0.36,0.00}{\textbf{#1}}}
\newcommand{\OtherTok}[1]{\textcolor[rgb]{0.56,0.35,0.01}{#1}}
\newcommand{\PreprocessorTok}[1]{\textcolor[rgb]{0.56,0.35,0.01}{\textit{#1}}}
\newcommand{\RegionMarkerTok}[1]{#1}
\newcommand{\SpecialCharTok}[1]{\textcolor[rgb]{0.00,0.00,0.00}{#1}}
\newcommand{\SpecialStringTok}[1]{\textcolor[rgb]{0.31,0.60,0.02}{#1}}
\newcommand{\StringTok}[1]{\textcolor[rgb]{0.31,0.60,0.02}{#1}}
\newcommand{\VariableTok}[1]{\textcolor[rgb]{0.00,0.00,0.00}{#1}}
\newcommand{\VerbatimStringTok}[1]{\textcolor[rgb]{0.31,0.60,0.02}{#1}}
\newcommand{\WarningTok}[1]{\textcolor[rgb]{0.56,0.35,0.01}{\textbf{\textit{#1}}}}
\usepackage{graphicx,grffile}
\makeatletter
\def\maxwidth{\ifdim\Gin@nat@width>\linewidth\linewidth\else\Gin@nat@width\fi}
\def\maxheight{\ifdim\Gin@nat@height>\textheight\textheight\else\Gin@nat@height\fi}
\makeatother
% Scale images if necessary, so that they will not overflow the page
% margins by default, and it is still possible to overwrite the defaults
% using explicit options in \includegraphics[width, height, ...]{}
\setkeys{Gin}{width=\maxwidth,height=\maxheight,keepaspectratio}
\setlength{\emergencystretch}{3em}  % prevent overfull lines
\providecommand{\tightlist}{%
  \setlength{\itemsep}{0pt}\setlength{\parskip}{0pt}}
\setcounter{secnumdepth}{-2}
% Redefines (sub)paragraphs to behave more like sections
\ifx\paragraph\undefined\else
  \let\oldparagraph\paragraph
  \renewcommand{\paragraph}[1]{\oldparagraph{#1}\mbox{}}
\fi
\ifx\subparagraph\undefined\else
  \let\oldsubparagraph\subparagraph
  \renewcommand{\subparagraph}[1]{\oldsubparagraph{#1}\mbox{}}
\fi

% set default figure placement to htbp
\makeatletter
\def\fps@figure{htbp}
\makeatother


\author{}
\date{\vspace{-2.5em}}

\begin{document}

\hypertarget{chapter-02---predicting-algae-blooms}{%
\section{Chapter 02 - Predicting Algae
Blooms}\label{chapter-02---predicting-algae-blooms}}

O objetivo deste estudo é realizar a predição de expansão de algas em
rios da Europa.

Foram coletadas diversas amostras de água em rios da Europa, durante
aproximadamente um ano. Foram medidas diferentes propriedades químicas e
a frequêncida de ocorrência de 7 tipos prejudiciais de algas. Outros
dados foram, ainda, incluídos, como estação do ano, dimensão e
velocidade do rio.

Como a coleta e análise de propriedades químicas do rio pode ser feita
de forma automática, rápida e barata e a análise biológica para
identificar algas envolve pessoal altamente treinando, além de ser cara
e lenta, a obtenção de modelos de predição do crescimento das algas a
partir de parâmetros químicos é bastante oportuna.

\hypertarget{os-dados}{%
\subsection{Os dados}\label{os-dados}}

Existem 2 datasets para esta análise. O primeiro possui 200 amostras com
as seguintes variáveis:

\begin{itemize}
\tightlist
\item
  Estação do ano;
\item
  Tamanho do rio;
\item
  Velocidade do rio;
\item
  8 parâmetros químicos:

  \begin{itemize}
  \tightlist
  \item
    Valor máximo do pH;
  \item
    Valor mínimo de Oxigênio (O2);
  \item
    Valor máximo de Cloro (Cl);
  \item
    Valor médio de Nitratos (NO-3);
  \item
    Valor médio de Amônia (NH+4);
  \item
    Média de Ortofosfato (PO3-4);
  \item
    Média de Fosfaro (PO4); e,
  \item
    Média de Clorofila.
  \end{itemize}
\item
  Associado a estes dados há a frequência de 7 algas prejudiciais.
\end{itemize}

O segundo dataset possui 140 amostras com os dados acima, porém sem as
frequências dos 7 tipos de algas prejudiciais. Nosso objetivo é fornecer
um modelo que permita prever estes valores.

\hypertarget{instalando-o-pacote-do-livro}{%
\subsection{Instalando o pacote do
livro}\label{instalando-o-pacote-do-livro}}

Para a primeira execução do código, retire o ``\#'' da linha abaixo:

\begin{Shaded}
\begin{Highlighting}[]
\CommentTok{# install.packages('DMwR')}
\end{Highlighting}
\end{Shaded}

\hypertarget{carregando-e-verificando-os-dados}{%
\subsection{Carregando e verificando os
dados}\label{carregando-e-verificando-os-dados}}

\begin{Shaded}
\begin{Highlighting}[]
\KeywordTok{library}\NormalTok{(DMwR)}
\end{Highlighting}
\end{Shaded}

\begin{verbatim}
## Warning: package 'DMwR' was built under R version 3.6.3
\end{verbatim}

\begin{verbatim}
## Loading required package: lattice
\end{verbatim}

\begin{verbatim}
## Loading required package: grid
\end{verbatim}

\begin{verbatim}
## Registered S3 method overwritten by 'quantmod':
##   method            from
##   as.zoo.data.frame zoo
\end{verbatim}

\begin{Shaded}
\begin{Highlighting}[]
\KeywordTok{str}\NormalTok{(algae)}
\end{Highlighting}
\end{Shaded}

\begin{verbatim}
## 'data.frame':    200 obs. of  18 variables:
##  $ season: Factor w/ 4 levels "autumn","spring",..: 4 2 1 2 1 4 3 1 4 4 ...
##  $ size  : Factor w/ 3 levels "large","medium",..: 3 3 3 3 3 3 3 3 3 3 ...
##  $ speed : Factor w/ 3 levels "high","low","medium": 3 3 3 3 3 1 1 1 3 1 ...
##  $ mxPH  : num  8 8.35 8.1 8.07 8.06 8.25 8.15 8.05 8.7 7.93 ...
##  $ mnO2  : num  9.8 8 11.4 4.8 9 13.1 10.3 10.6 3.4 9.9 ...
##  $ Cl    : num  60.8 57.8 40 77.4 55.4 ...
##  $ NO3   : num  6.24 1.29 5.33 2.3 10.42 ...
##  $ NH4   : num  578 370 346.7 98.2 233.7 ...
##  $ oPO4  : num  105 428.8 125.7 61.2 58.2 ...
##  $ PO4   : num  170 558.8 187.1 138.7 97.6 ...
##  $ Chla  : num  50 1.3 15.6 1.4 10.5 ...
##  $ a1    : num  0 1.4 3.3 3.1 9.2 15.1 2.4 18.2 25.4 17 ...
##  $ a2    : num  0 7.6 53.6 41 2.9 14.6 1.2 1.6 5.4 0 ...
##  $ a3    : num  0 4.8 1.9 18.9 7.5 1.4 3.2 0 2.5 0 ...
##  $ a4    : num  0 1.9 0 0 0 0 3.9 0 0 2.9 ...
##  $ a5    : num  34.2 6.7 0 1.4 7.5 22.5 5.8 5.5 0 0 ...
##  $ a6    : num  8.3 0 0 0 4.1 12.6 6.8 8.7 0 0 ...
##  $ a7    : num  0 2.1 9.7 1.4 1 2.9 0 0 0 1.7 ...
\end{verbatim}

\begin{Shaded}
\begin{Highlighting}[]
\KeywordTok{summary}\NormalTok{(algae)}
\end{Highlighting}
\end{Shaded}

\begin{verbatim}
##     season       size       speed         mxPH            mnO2       
##  autumn:40   large :45   high  :84   Min.   :5.600   Min.   : 1.500  
##  spring:53   medium:84   low   :33   1st Qu.:7.700   1st Qu.: 7.725  
##  summer:45   small :71   medium:83   Median :8.060   Median : 9.800  
##  winter:62                           Mean   :8.012   Mean   : 9.118  
##                                      3rd Qu.:8.400   3rd Qu.:10.800  
##                                      Max.   :9.700   Max.   :13.400  
##                                      NA's   :1       NA's   :2       
##        Cl               NO3              NH4                oPO4       
##  Min.   :  0.222   Min.   : 0.050   Min.   :    5.00   Min.   :  1.00  
##  1st Qu.: 10.981   1st Qu.: 1.296   1st Qu.:   38.33   1st Qu.: 15.70  
##  Median : 32.730   Median : 2.675   Median :  103.17   Median : 40.15  
##  Mean   : 43.636   Mean   : 3.282   Mean   :  501.30   Mean   : 73.59  
##  3rd Qu.: 57.824   3rd Qu.: 4.446   3rd Qu.:  226.95   3rd Qu.: 99.33  
##  Max.   :391.500   Max.   :45.650   Max.   :24064.00   Max.   :564.60  
##  NA's   :10        NA's   :2        NA's   :2          NA's   :2       
##       PO4              Chla               a1              a2        
##  Min.   :  1.00   Min.   :  0.200   Min.   : 0.00   Min.   : 0.000  
##  1st Qu.: 41.38   1st Qu.:  2.000   1st Qu.: 1.50   1st Qu.: 0.000  
##  Median :103.29   Median :  5.475   Median : 6.95   Median : 3.000  
##  Mean   :137.88   Mean   : 13.971   Mean   :16.92   Mean   : 7.458  
##  3rd Qu.:213.75   3rd Qu.: 18.308   3rd Qu.:24.80   3rd Qu.:11.375  
##  Max.   :771.60   Max.   :110.456   Max.   :89.80   Max.   :72.600  
##  NA's   :2        NA's   :12                                        
##        a3               a4               a5               a6        
##  Min.   : 0.000   Min.   : 0.000   Min.   : 0.000   Min.   : 0.000  
##  1st Qu.: 0.000   1st Qu.: 0.000   1st Qu.: 0.000   1st Qu.: 0.000  
##  Median : 1.550   Median : 0.000   Median : 1.900   Median : 0.000  
##  Mean   : 4.309   Mean   : 1.992   Mean   : 5.064   Mean   : 5.964  
##  3rd Qu.: 4.925   3rd Qu.: 2.400   3rd Qu.: 7.500   3rd Qu.: 6.925  
##  Max.   :42.800   Max.   :44.600   Max.   :44.400   Max.   :77.600  
##                                                                     
##        a7        
##  Min.   : 0.000  
##  1st Qu.: 0.000  
##  Median : 1.000  
##  Mean   : 2.495  
##  3rd Qu.: 2.400  
##  Max.   :31.600  
## 
\end{verbatim}

\begin{Shaded}
\begin{Highlighting}[]
\KeywordTok{head}\NormalTok{(algae)}
\end{Highlighting}
\end{Shaded}

\begin{verbatim}
##   season  size  speed mxPH mnO2     Cl    NO3     NH4    oPO4     PO4 Chla   a1
## 1 winter small medium 8.00  9.8 60.800  6.238 578.000 105.000 170.000 50.0  0.0
## 2 spring small medium 8.35  8.0 57.750  1.288 370.000 428.750 558.750  1.3  1.4
## 3 autumn small medium 8.10 11.4 40.020  5.330 346.667 125.667 187.057 15.6  3.3
## 4 spring small medium 8.07  4.8 77.364  2.302  98.182  61.182 138.700  1.4  3.1
## 5 autumn small medium 8.06  9.0 55.350 10.416 233.700  58.222  97.580 10.5  9.2
## 6 winter small   high 8.25 13.1 65.750  9.248 430.000  18.250  56.667 28.4 15.1
##     a2   a3  a4   a5   a6  a7
## 1  0.0  0.0 0.0 34.2  8.3 0.0
## 2  7.6  4.8 1.9  6.7  0.0 2.1
## 3 53.6  1.9 0.0  0.0  0.0 9.7
## 4 41.0 18.9 0.0  1.4  0.0 1.4
## 5  2.9  7.5 0.0  7.5  4.1 1.0
## 6 14.6  1.4 0.0 22.5 12.6 2.9
\end{verbatim}

Vamos obter abaixo um histograma de ``mxPH''. O parâmetro ``prob = T''
nos dá a probabilidade de cada intervalo de valor, caso contrário
teríamos a contagem.

\begin{Shaded}
\begin{Highlighting}[]
\KeywordTok{hist}\NormalTok{(algae}\OperatorTok{$}\NormalTok{mxPH, }\DataTypeTok{prob =}\NormalTok{ T)}
\end{Highlighting}
\end{Shaded}

\includegraphics{Chapter02_files/figure-latex/unnamed-chunk-4-1.pdf}

\begin{Shaded}
\begin{Highlighting}[]
\KeywordTok{hist}\NormalTok{(algae}\OperatorTok{$}\NormalTok{mxPH)}
\end{Highlighting}
\end{Shaded}

\includegraphics{Chapter02_files/figure-latex/unnamed-chunk-5-1.pdf} O
histograma abaixo, com o QQ Plot permite verificar a normalidade dos
dados.

Q Q Plots (Quantile-Quantile plots) are plots of two quantiles against
each other. A quantile is a fraction where certain values fall below
that quantile. For example, the median is a quantile where 50\% of the
data fall below that point and 50\% lie above it. The purpose of Q Q
plots is to find out if two sets of data come from the same
distribution. A 45 degree angle is plotted on the Q Q plot; if the two
data sets come from a common distribution, the points will fall on that
reference line.

Instale o pacote abaixo, se necessário.

\begin{Shaded}
\begin{Highlighting}[]
\CommentTok{# install.packages('car')}
\end{Highlighting}
\end{Shaded}

\begin{Shaded}
\begin{Highlighting}[]
\KeywordTok{library}\NormalTok{(car)}
\end{Highlighting}
\end{Shaded}

\begin{verbatim}
## Warning: package 'car' was built under R version 3.6.3
\end{verbatim}

\begin{verbatim}
## Loading required package: carData
\end{verbatim}

\begin{verbatim}
## Warning: package 'carData' was built under R version 3.6.3
\end{verbatim}

\begin{Shaded}
\begin{Highlighting}[]
\KeywordTok{par}\NormalTok{(}\DataTypeTok{mfrow=}\KeywordTok{c}\NormalTok{(}\DecValTok{1}\NormalTok{,}\DecValTok{2}\NormalTok{)) }\CommentTok{# dividindo o painel do gráfico por 1 linha e 2 colunas}
\KeywordTok{hist}\NormalTok{(algae}\OperatorTok{$}\NormalTok{mxPH, }\DataTypeTok{prob =}\NormalTok{ T, }\DataTypeTok{xlab =} \StringTok{""}\NormalTok{,}
     \DataTypeTok{main =} \StringTok{"Histograma do valor máximo de pH"}\NormalTok{, }\DataTypeTok{ylim =} \DecValTok{0}\OperatorTok{:}\DecValTok{1}\NormalTok{)}
\KeywordTok{lines}\NormalTok{(}\KeywordTok{density}\NormalTok{(algae}\OperatorTok{$}\NormalTok{mxPH,}\DataTypeTok{na.rm =}\NormalTok{ T))}
\KeywordTok{rug}\NormalTok{(}\KeywordTok{jitter}\NormalTok{(algae}\OperatorTok{$}\NormalTok{mxPH))}
\KeywordTok{qqPlot}\NormalTok{(algae}\OperatorTok{$}\NormalTok{mxPH, }\DataTypeTok{main =} \StringTok{"Normal QQ Plot de máximo de pH"}\NormalTok{)}
\end{Highlighting}
\end{Shaded}

\includegraphics{Chapter02_files/figure-latex/unnamed-chunk-7-1.pdf}

\begin{verbatim}
## [1] 56 57
\end{verbatim}

\begin{Shaded}
\begin{Highlighting}[]
\KeywordTok{par}\NormalTok{(}\DataTypeTok{mfrow=}\KeywordTok{c}\NormalTok{(}\DecValTok{1}\NormalTok{,}\DecValTok{1}\NormalTok{))}
\end{Highlighting}
\end{Shaded}

A parte de baixo do histograna mostra a distribuição dos dados. Ali é
possível ver dois valores baixos muito apartados dos outros
valores\ldots{} Possivelmente outliers.

o QQ Plot mostra a distribuição normal (linha contínua azul) e um
intervalo de confiança de 95\% (linha tracejada azul). Note que alguns
valores baixos estão fora do intervalo de confiança da distribuição
normal.

Outra possibilidade é criar um Boxplot, que permite uma rápida
visualização da distribuição dos dados, a posição da mediana, os quartis
e eventuais outliers.

\begin{Shaded}
\begin{Highlighting}[]
\KeywordTok{boxplot}\NormalTok{(algae}\OperatorTok{$}\NormalTok{oPO4, }\DataTypeTok{ylab =} \StringTok{"Ortofosfato - oP04"}\NormalTok{)}
\KeywordTok{rug}\NormalTok{(}\KeywordTok{jitter}\NormalTok{(algae}\OperatorTok{$}\NormalTok{oPO4), }\DataTypeTok{side =} \DecValTok{2}\NormalTok{) }\CommentTok{# "rug" gera os "risquinhos" ao lado do gráfico, mostram o espalhamento dos dados; "jitter" melhora a visualização destes risquinhos, evitando a sobreposição deles; side = 2 é o eixo Y.}
\KeywordTok{abline}\NormalTok{(}\DataTypeTok{h =} \KeywordTok{mean}\NormalTok{(algae}\OperatorTok{$}\NormalTok{oPO4, }\DataTypeTok{na.rm =}\NormalTok{ T), }\DataTypeTok{lty =} \DecValTok{2}\NormalTok{) }\CommentTok{# essa linha contendo a média permite a comparação com a mediana, no caso, o descasamento das duas mostra o efeito dos outliers na amostra.}
\end{Highlighting}
\end{Shaded}

\includegraphics{Chapter02_files/figure-latex/unnamed-chunk-8-1.pdf}

\end{document}
